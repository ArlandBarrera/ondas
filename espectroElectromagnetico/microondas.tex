Abarcan frecuancias entre 300 MHz y 300 GHz. Se utilizan en los hornos microondas, comunicaciones inalámbricas, radares y sistemas de satélite.

\subsection{Horno microondas}

Las microondas son emitidas por un dispositivo llamado magnetrón. Estas chocan con las paredes metalicas del horno y crean ondas de tipo estacionaria, por tanto hay zonas más calientes que otras, para asegurar una cocción uniforme se utiliza un plato rotatorio. La componente del campo eléctrico de la onda electromagnética interactua con las moléculas de agua (H2O) en los alimentos. Estas moléculas son polares, el hidrógeno es el polo positivo y el oxígeno el negativo, de modo que giran con el campo eléctrico. Esto resulta en una vibración y al chocar entre sí generan calor.

\begin{figure}[H]
  \centering
  \includegraphics[scale=0.5]{imagenes/microwave.png}
  \caption{1955, microondas montado en la pared, Tappan RL-1\cite{i3especmicrowave}}
\end{figure}
