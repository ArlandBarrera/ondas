La energía se denota la letra $E$ mayúscula. Su unidad en el \textbf{Sistema Internacional} es el \textbf{Joule} $(J)$.

Su cálculo depende de la naturaleza de la onda.

Para la energía de un fotón se considera la \textbf{constante de Planck} $h$, cuya unidad es Joule-segundo $(Js)$. Su valor aproxmiado es el siguiente:

\begin{listequbox}
  { h = 6.626\text{x}10^{-34} Js}{equplanckconst}{Constante de Planck $h$}
\end{listequbox}

Esta contante relaciona la energía de un fotón con su frecuencia. A continuación se muestra la expresión:

\begin{listequbox}
  {E=hf}{equengfot}{Energía de un fotón}
\end{listequbox}

La unidad resultante es Joule $(J)$.

\[ E = Js(Hz) = Js\left(\dfrac{1}{s}\right) = J \]

En ciertos contextos para medir la energía se utiliza el Electronvolt $(eV)$ en lugar del Joule. La relación entre ambas unidades se muestra en la siguiente equivalencia:

\begin{listequbox}
  {1 eV = 1.602\text{x}10^{-19} J}{equelvjul}{Relación entre Electronvolt $(eV)$ y Joule $(J)$}
\end{listequbox}

De la expresión \ref{equengfot} se puede inferir que \textbf{la energía es mayor cuando la frecuencia aumenta}.
