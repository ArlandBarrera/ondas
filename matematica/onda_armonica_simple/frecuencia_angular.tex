Traslación en radianes por unidad de tiempo. Dicho de otra forma, cuanto se mueve la onda, en términos de radianes, en un periodo. Se representa mediante la letra griega omega minúscula $(\omega)$.

Se relaciona con el periodo de la siguiente forma:

\begin{listequbox}
  {\omega=\dfrac{2\pi}{T}}{equfrecang}{Frecuencia angular y periodo}
\end{listequbox}

De esto se infiere que la unidad de $\omega$ es radián por segundo $\left(\dfrac{\pi}{s}\right)$.

De la ecuación \ref{equfrecang} se desprenden las siguientes:

\[\boxed{
  T = \dfrac{2\pi}{\omega}
}\]

Cade recordar la ecuación \ref{equfrecuencia}, dado que la frecuencia angular también se relaciona con la frecuencia. Por tanto se obtiene:

\begin{listequbox}
  {\omega = f2\pi}{equfrecangfrec}{Frecuencia angular y frecuencia}
\end{listequbox}
