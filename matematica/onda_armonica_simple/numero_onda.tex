Traslación en radianes por unidad de longitud. Dicho de otra forma, cuanto se mueve la onda, en términos de radianes, en una longitud de onda. Se representa mediante la letra $k$ minúscula.

Se relaciona con la longitud de onda de la siguiente forma:

\begin{listequbox}
  {k=\dfrac{2\pi}{\lambda}}{equnumonda}{Número de onda}
\end{listequbox}

De esto se infiere que la unidad de $k$ es radián por metro $\left(\dfrac{\pi}{m}\right)$.

De la ecuación \ref{equnumonda} se desprenden las siguientes:

\[\boxed{
  \lambda = \dfrac{2\pi}{k}
}\]

Obtener la relación inversa resulta conveniente para el cálculo de la velocidad.

\begin{listequbox}
  {\dfrac{1}{k} = \dfrac{\lambda}{2\pi}}{equinvnumlng}{Relación inversa entre número de onda y longitud de onda}
\end{listequbox}
