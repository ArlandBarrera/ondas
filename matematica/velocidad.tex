Hay que considerar la ecuación fundamental del Movimiento Rectilineo Uniforme (MRU):

\[\boxed{
  v = \dfrac{d}{t}
}\]

donde las variables indican:

\begin{itemize}
  \item $v$: velocidad $\left(\dfrac{m}{s}\right)$.
  \item $d$: distancia ($m$).
  \item $t$: tiempo ($s$).
\end{itemize}

Adaptando esta ecuación al contexto de una onda, hay que considerar la longitud de onda ($\lambda$) y periodo ($T$).

\[\boxed{
  v = \dfrac{\lambda}{T}
}\]

Haciendo uso de la ecuación \ref{equfrecuencia} se obtiene la siguiente expresión:

\begin{listequbox}
  {v = \lambda f}{equvelonda}{Velocidad de una onda}
\end{listequbox}

La velocidad también se puede expresar en función del número de onda y frecuencia angular. Para ello para hay que considerar las ecuaciones \ref{equinvnumlng} y \ref{equfrecangfrec}. Para aplicar esas expresiones $2\pi$ tiene que estar presente, para lograr eso se multiplica la ecuacion por $\dfrac{2\pi}{2\pi}$, que equivale a $1$ y no altera la ecuación.

\begin{align*}
  v &= \lambda f\left(\dfrac{2\pi}{2\pi}\right) \\
  v &= \left(\dfrac{\lambda}{2\pi}\right)\left(f2\pi\right)
\end{align*}

Se obtiene:

\begin{listequbox}
  {v = \dfrac{\omega}{k}}{equvelnumfrecang}{Velocidad en función del número de onda y frecuencia angular}
\end{listequbox}

En ciertos contextos, como en el estudio de fotones, se considera la velocidad de la luz en el vacío $(c)$.

\begin{listequbox}
  {c = 3\text{x}10^8 \dfrac{m}{s}}{equvelluz}{Velocidad de la luz en el vacío $(c)$}
\end{listequbox}

De la expresión \ref{equvelonda} se puede extraer que \textbf{la frecuencia y la longitud de onda tienen una relación inversa}, cuando aumenta una disminuye la otra. Y también se concluye que \textbf{la velocidad tiene una relación directa con la frecuencia}, una aumenta cuando la otra aumenta. En sentido contrario se infiere que \textbf{la velocidad tiene una relación inversa con la longitud de onda}, una aumenta cuando disminuye la otra.
