% ------------------------------
% Automatizar llamar [doc]umento [ex]terno -> \exdoc
% ------------------------------
\newcommand{\exdoc}[1]{
  \externaldocument{#1/ref-#1}
}

% [It]erar temas con [doc]umento [ex]terno -> \itexdoc
%
\newcommand{\itexdoc}[1]{
  \forcsvlist{\exdoc}{#1}
}

% Crear referencias a los documentos de los temas
%
\itexdoc{\temas}

% ------------------------------
% Automatizar crear hipervinculo -> \link
% ------------------------------
% NOTE: Se crea el hipervinculo partir de la referencia al documento externo
% por eso es importante definir el documento externo antes de esto
%
\newcommand{\link}[1]{
  \expandafter\newcommand\csname #1\endcsname{
    \unskip\hyperref[lbl-#1]{#1}\ignorespaces
  }
}
%
% % [It]erar temas con \link -> \itlink
% %
\newcommand{\itlink}[1]{
  \forcsvlist{\link}{#1}
}
%
% % Crear hipervinculos
% % Temas
% \expandafter\itlink\expandafter{\temas}
%
% % Subtemas
\expandafter\itlink\expandafter{\subtemas}

% Hipervinculos personalizados manuales
%
\newcommand{\EspectroElectromagnetico}{
  \unskip\hyperref[lbl-espectroElectromagnetico]{espectro electromagnético}\ignorespaces
}
\newcommand{\electromagnetica}{
  \unskip\hyperref[lbl-electromagnetica]{electromagnética}\ignorespaces
}
\newcommand{\mecanica}{
  \unskip\hyperref[lbl-mecanica]{mecánica}\ignorespaces
}
